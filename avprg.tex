\documentclass[12pt]{Article}
\usepackage[ngerman]{babel}
\usepackage[utf8]{inputenc}
\usepackage[round,authoryear]{natbib}
\usepackage[hyphens]{url}
\usepackage{amssymb}
\usepackage{hyperref}
\usepackage{float}
\usepackage[]{graphicx}
\usepackage[all]{hypcap}

\usepackage{caption}
\usepackage{subcaption}
\usepackage{titling}
%\pagenumbering{gobble}
\title{AVPRG Konzept \glqq Abakus\grqq}
\author{ Stefan Schäfers (2175460), Julian Roosch (2203745)}
\begin{document}
\maketitle
\begin{abstract}
Ein Abakus ist ein mechanisches Rechenhilfsmittel, welches Kugeln, oft aus Holz oder Glas, verwendet, die auf einer horizontalen Strecke verschoben werden können. Je nach Ausführung wird auch die Bezeichnung Zählrahmen verwendet.

In unserem Projekt soll ein Abakus zur Steuerung von Software und Audiosignalen verwendet werden.
\end{abstract}
\section{Beschreibung aus Nutzersicht}
%nutzer verschiebt kugeln auf abakus
%je eine reihe abbildung eines schiebereglers zum steuern eines bestimmten parameters
%verschiebt werte auf virtuellem interface -> tangable schieberegler
%controller für sound erzeugungsprogramm (midicontroller)
%Bild Versuchsaufbau
%
Ein Nutzer kann an einer Art Abakus kleine Holzkugeln verschieben. Jede Reihe stellt dabei einen eigenen Parameter dar, wobei die Kugel als Art \glqq tangable Schieberegler\grqq  fungiert. Durch das verschieben der Holzkugeln werden die Werte des Interfaces verändert -- so entsteht aus dem Abakus ein kleiner Midicontroller.
%
Ein Nutzer soll an einem Abakus ähnelndem Objekt farbige Kugeln auf vertikal übereinander angeordneten Stangen frei bewegen können. Je eine Kugel beschreibt in Relation zu ihrer Position auf der jeweiligen Stange einen Wert. Dieser Wert soll anschließend an eine Schnittstelle weitergeleitet werden, damit mit den Kugeln ein Interface gesteuert werden kann. Jede Kugel stellt somit eine Art \glqq tangable Schieberegler\grqq dar.


\section{Technisches Konzept}
%abakus ähnliches konstrukt wird erstellt (lackierte holzkugeln verschiebbar auf metallstangen o.ä.)
%farbige kugeln übereinander angeordnet
%gut beleuchtet, weißer hintergrund
%kamera erfasst kugeln
%erfassung der positionen durch opencv (filterung der signale)
%auswertung der werte -> ggf skalierung wie in oscillator projekt
%weiterleitung an software (zb oscillator) / senden eines midi wertes

%bau eines abakus
Jede auf dem Abakus befindliche farbige Kugel wird von einer Kamera erfasst, die sich unmittelbar vor dem Abakus positioniert wird. Der Abakus wird dabei gut beleuchtet, damit die Kamera möglichst kontrastreiche Bilddaten erhält. Um weitere Fehlerquellen zu vermeiden wird der von der Kamera erfasste Bereich einfarbig Verklediet (Karton, Papier, Pappe o.ä.), damit die Farberkennung außerhalb des gewünschten Bereiches reduziert wird (Erkennung der Farbe(n) der Kugeln in der Umgebung vermeiden). Die von der Kamera aufgezeichneten Einzelbilder werden in einem Analyseprogramm mit Hilfe der Programmbibliothek \glqq OpenCV\grqq ausgewertet. Hier findet die Objekterkennung der Kugeln anhand ihrer Farbe statt. Wie bei einem Schieberegler bestimmt die Position des \glqq Schiebeelements\grqq den Wert des jeweiligen Reglers. Jede Reihe beschreibt dabei den Wert eine bestimmten Variable. Mit dem horizontalen Bewegen einer Kugel von links nach rechts soll dieser Wert von 0\% bis 100\% eingestellt werden können. Abhängig von der Variable, die mit dem Regler angesteuert werden soll, wird dieser wert auf einen bestimmten Wertebereich abgeblidet und/oder skaliert. Dieser Wert wird aus der ermittelten Position der Kugel generiert und wird anschließend entweder über einen korrespondierenden MIDI-Wert an eine DAW geschickt oder direkt an die variablen Einstellungsmöglichkeiten des Synthesizers aus der AVPRG Vorlesung weitergeleitet.



\end{document}