\documentclass[12pt]{article}
\usepackage[ngerman]{babel}
\usepackage[utf8]{inputenc}
\usepackage[round,authoryear]{natbib}
\usepackage[hyphens]{url}
\usepackage{amssymb}
\usepackage{hyperref}
\usepackage{float}
\usepackage[]{graphicx}
\usepackage[all]{hypcap}
\usepackage{titlepic}
\usepackage{caption}
\usepackage{subcaption}
\usepackage{titling}
%\pagenumbering{gobble}
\title{Object Tracking}
\setlength{\droptitle}{-0em}
\author{Stefan Schäfers (2175460), Julian Roosch (2203745)}
\begin{document}
\maketitle
%Object Tracking in einem Maskierten Bild, Pseudocode, schlechte Laufzeit, optimierung durch methoden die bereits in opencv implementiert sind
\begin{enumerate}
\item So lange nicht alle Pixel durchlaufen, das Bild Zeile für Zeile von links nach rechts durchsuchen
\item Wenn ein Weißes Pixel gefunden wird, schwarz färben; x- und y-Wert des Pixels auf Variablen addieren, Anzahl um eins erhöhen
\item So lange \glqq unbesuchte\grqq{} Pixel existieren, finde alle Nachbarpixel und wiederhole Schritt 2
\item Wenn die Anzahl der Pixel eine Mindestgröße überschreitet wurde ein Objekt erkannt
\item Sonst war es nur Rauschen oder sonstige Fehlerquellen
\end{enumerate}

\begin{enumerate}
\item Hole Inputimage von Kamera
\item Konvertieren des Bildes in den HSV-Farbraum für bessere Thresholds
\item Anwendung von medianBlur, um Rauschen aus dem Ausgangsbild zu entfernen
\item Maskieren des Bildes über Thresholds
\item Anwendung von medianBlur, um Rauschen aus dem maskierten Bild zu entfernen
\item Anwendung von errode und dillate, um weiteres (weißes) Rauschen zu entfernen und um Objekte zu schließen
\item Finden der Konturen durch findContours
\item Anhand der Größe der Rückgabe erkennen, ob die korrekte Anzahl an Objekten mit der gewünschten Größe gefunden werden
\item Ist dies der Fall, Berechnung der Zentren der Objekte über moments
\end{enumerate}

\end{document}